\section{How This Book Was Written}
Junk DNA corresponds to the 98.5\% part of the human genome that has no
genetic function. Thierry Bardini has explored in Junkware (Minnesota
Press, 2010) the idea that {\textquotedbl}most of culture and nature,
including humans, is composed of useless, but always potentially
recyclable, material otherwise known as
{\textquoteleft}junk.{\textquoteright} {\textquotedbl}\newline
Are we made of junk ? Do we have become literaly junk ? \newline
\newline
Inspired by Bardini{\textquotesingle}s book, a group of french \&
chinese artists has choosen to interprete this concept through an
eponym installation : a speculative machine that uses mathematical
modeling and data analysis to generate futuristic objects from DNA and
brainwaves. \newline
\newline
The opportunity to present Junkware installation in Shangha\"i (Maker
Carnival, october 2014) is combined with the organisation of a textbook
sprint aiming at sharing, collecting and editing materials - ideas as
well as data - to observe, critic and speculate on our
own{\textquotesingle}s practice towards junk : the one that is produced
and the one~ reused. 

Your junk is a treasure for someone else. \newline
\newline
This book is a recollection of contributions from the Junk Book Venture
workshop organized at One Space on 15\&16th october in Shangha\"i
(Chapter One) ; and as a metaphor of the Junk concept itself, it
includes a great remix of archives (data, text, visuals) produced from
november 2013 to october 2014 to prepare Junkware Installation (Chapter
Two).
