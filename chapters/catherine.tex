\section{Data-Literature}

First time I encountered the term \textit{data-literature}was in an article written by the french web-journalist Sabine Blanc untitled {\textquotedbl} \href{http://owni.fr/2011/03/18/de-la-datalitterature-dans-le-9-3/}{De
la data-litt\'erature dans le 93}{\textquotedbl}. She shared the story
of {\textquotedbl}Traques Traces{\textquotedbl} project reflecting on
digital identity and based on statistics and one-line data, developed
by the writer C\'ecile Portier together with teenagers in an high
school in Paris{\textquotesingle}suburbs where sha was hosted in
residency.

Perhaps, Sabine Blanc coined the term data-literature (I could hardly
find other references later on the web) without noticing it. At that
time, she worked for OWNI (acronym for Non Identified Web Objects) an
information website initiated in 2009 which gained its reputation based
on a solid data-journalism, meantime introducing in France this new way
of thinking \& doing in the field of journalism.

\href{https://en.wikipedia.org/wiki/Data_journalism}{\textit{Data
journalism}}\textit{ is a journalism specialty reflecting
the increased role that numerical data is used in the production and
distribution of information in the digital era. It reflects the
increased interaction between content producers (journalist) and
several other fields such as design, computer science and statistics}
as defined by Wikipedians.~

If Sabine Blanc never proposed a definition in her article, I embbeded
the formula with me as it perfectly fit with the idea I have of my own
fictional playground. Especially for the novel I{\textquotesingle}m
currently writing. It has been convenient to present it under this
term, and so, it is maybe time for me now to come with a definition, or
at least, precision.

Data literature calls for fictional writing relying totally or partially
on sets of data. The narratives baseline explores one or various topics
with a guiding principle : to make those data speak in prose manner
implying a plot or not. Written in natural language and/or programing
language, it takes on multiple formats to perform the same work
(printed book, digital file, website, script, etc).

In 2012, I started an artistic literary project based on the life \&
work of the british novelist Anna Kavan (1901-1968). Originally
publishing under her first married name, Helen Ferguson, she adopted
the name Anna Kavan in 1939, not only as nom de plume but as her legal
identity.

I{\textquotesingle}ve been working meantime on a novel and on-line
artwork. In the novel, I choose to make
Kavan\textit{\textup{appears as an heroin and
Kavan-the-writer is also discussed by the different characters who
inhabit the novel. The whole background is also made out of my own
autofictional material as well as elements of data-literature}} (graph,
scripts, semantic vocabulary, dispersion plot analysis, etc). The
on-line artwork\textit{\textup{dialogues with the print
novel ; it is a website but it is not a replication of the novel. Two
chapters out of the five, will be accessible on-line while the main
part will be built out of data (blogposts, reviews, photographies,
bookcovers, fan art, quotes, tweets, etc) I{\textquotesingle}ve been
collecting in the last four years. Something like a semantic and
fictional active archives. }}

\textit{\textup{But nothing like a biography for all these
reasons : A biography would mean to put in front again her sensational
life and I was much more interested by the autobiographical puzzle she
elaborated, her literary project let{\textquotesingle}s say. A
biographer needs to say everything about his, her subject, often
sticking to a chronological lifeline and that{\textquotesingle}s hard
to keep such a line when you{\textquotesingle}re up to disruptive
narratives. And what is sad with biography is that we always know when
and how the book ends : with the death of the subject. And that would
be precisely when and how I wanted to start : Kavan would be already
dead but she would come back, I would recall her as a living heroin in
the fiction while other characters would tell us how
Kavan{\textquotesingle}s heritage lives in the present time. This
means, that unlike a biographer, I would consider all the web-folkore
and data I{\textquotesingle}ve been browsing, as a source, a minor
source, however a source to take into account, to play with.}}

\textit{\textup{That{\textquotesingle}s probably how I feel
Junkware echoes with my own practice : I{\textquotesingle}m writing
stories out of junk, digging the web, recollecting junk that embbeds
their own narratives and celebrate them as part of a fiction
territory.}}
