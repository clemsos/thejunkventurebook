\section{Anatomy of the workshop}

\subsection{An invitation}

On October 15, Professor Lafleche~from France and~MAKE+~from Shanghai
brought together 15 people from different backgrounds and cultures to
merge ideas to produce a book on the subject of
{\textquotedblleft}JUNK{\textquotedblright} at ONE~Space~. In this
workshop, scientists, artists and designers and professionals from
other fields have discussed together the changes in
today{\textquoteright}s and future{\textquoteright}s product design and
production.

In human DNA, only 1.5\% of DNA has genetic functions, the scientists
named the rest 98.5\% {\textquotedblleft}Junk{\textquotedblright},
because their functions have not been found. Canadian writer Thierry
Bardini defined {\textquotedblleft}Junk{\textquotedblright} in his book
{\textquotedblleft}Junkware{\textquotedblright} as this:
{\textquotedblleft}Almost all culture and nature, including ourselves,
are made of useless but recyclable materials. These materials are
called Junk{\textquotedblright}~In fact, our daily lives are surrounded
by useless products, the world is filled with Junk.~Are we really using
Junk to make things? Will we ever become Junk one day? Is Junk
completely useless or will they give us hope?

\subsection{A space }

ONE is a creative experimental space located in Shanghai Jingan
district. It is open to all company and creative individuals.
Participants are welcomed to take part in creative thinking process and
space experimentations through workshops, talks and exhibitions.

\subsection{A collaborative process}

Professor Lafleche and~MAKE+ have collected, shared and edited
information on Junk using text, image and visualized data and the
method of co-writing and annoting (English and Chinese) to discuss art,
design, innovation, technology, production, consumerism, patenting and
recycling in the future.

\subsection{A publishing platform}

Booktype is web-based software which means you do not install it on your
computer; rather, you access it through a web browser. It is Free
Software (licensed under the GNU Affero GPL), meaning that it can be
freely downloaded, re-used and customised. Booktype was originally
developed to facilitate the accelerated production models enabled by
book sprints. Hence the feature set matches the rapid pace of
publishing possible in the modern era of print-on-demand. Using
Booktype, print ready source (book formatted PDF) can be generated in a
few minutes, and then uploaded to your favourite print-on-demand
service.

\textbf{Thanks to all the participants !}

