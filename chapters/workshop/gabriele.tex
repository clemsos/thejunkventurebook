\section{*digital*media*junk*ware*lore*}
by Gabriele de Seta \url{http://paranom.asia} \\

\zh{*数 *媒 *废 *品 *俗 *}\\
\textbf{Shanghai mediations on Thierry Bardini{\textquotesingle}s}\textit{\textbf{Junkware}}



*junk*\newline
\zh{废物 (feiwu, garbage)} \textit{(feiwu}, garbage)\newline
{\textquotedbl}There is the rubbish we keep, which is junk, and the
rubbish we throw away, which is garbage{\textquotedbl}, Sydney Brenner
reportedly explains on page 49, inspiring Thierry
Bardini{\textquotesingle}s call for a semantics of discard. Of the many
questions elicited by the signature of *junk* - \textit{what
is junk?} Everything / \textit{when is junk?} Now, and more
to come / \textit{where is junk?} Everywhere /
\textit{who is junk? }Us, and them / \textit{how
is junk?} Good and bad and useful and useless, it depends, yet
it{\textquotesingle}s 98.5\% of everything - perhaps the most sensible
question to ask today remains \textit{so what?},
or,\textit{ why does it matter?}, or, \textit{why
does junk matter?} and \textit{why does matter junk?}

*ware*\newline
\zh{无用的}(\textit{wuyong de}, useless)\newline
A materialist approach to junk and a junkist approach to matter seem the
most useful moves to derive from Bardini{\textquotesingle}s
\textit{portmanteau} *junk*ware*: stuff that is the Other of
stuff, in time and place and ownership and use, things that are
temporarily thrown in a limbo of less-thingness and uncertainty,
accumulating in heaps of problematic matter, losing the boundaries
which segment them in individual objects, rewinding back the process of
individuation by piling up upon other junk. The next step leads into
speculative junkonomics: a recuperated copy of Marx{\textquotesingle}s
writings encoded in the wrong character set postulates the W-J-W and
J-W-J cycles, from ware to junk to ware, and from junk to ware to junk,
the accumulation of capital sustained by use value superseded by the
self-sustaining accretion of use-less value in stacks of [54C1?]
\textit{pin}, {\textquotesingle}ware{\textquotesingle}.

*digital*\newline
\zh{垃圾邮件} (\textit{laji youjian}, junk
mail)\newline
The fabled Internet of Things is in fact merely the polished 1.5\%
portrayed by the aseptic photos of data centers and the colorful
network diagrams appearing on sleek PowerPoint presentations at
international industry talks. What about the non-signifying 98.5\%,
then? Where is The Internet of Junk? Who collects and sorts out
*digital*junk*ware*? What other forms does junk take in our digital and
already postdigital times, besides the obvious junk mail already made
almost invisible by the perfected algorithms of our filters and the
pop-ups preemptively killed off by our AdBlock plug-ins? How does this
digital junk get repurposed into digital ware and when does digital
ware goes back to being digital junk? Is even junk a fitting metaphor
for the platitude that user-generated content is?

*media*\newline
\zh{信息毒品} (\textit{xinxi dupin},
information drug)\newline
Bardini{\textquotesingle}s writing is irremediably double-helixed,
knitted along the polymer bases of its own DNA, as it proceeds by a
sort of augmented dialectics set in motion by hidden third terms -
\textit{media}. DNA is made of genes, but genes are 98.5\%
junk, and junk matters; culture is made of memes, and memes are just
like genes, so 98.5\% of culture is junk, and it matters too. The jump
from these parellels into the hidden third of hypervirality is
legitimized through the figure of the loop, which becomes the
organizing aesthetic trope of our times after the deleuzian fold. In
hyperviral culture, the augmented dialectic loop continues, memes are
the genes of culture, and memes work like mind viruses, so that
{\textquotedbl}a virus is essentially junk code, and our hyperviral
culture is indeed a junk culture{\textquotedbl} (p. 189). The viral
metaphor mediating between memetics and genetics pierces through half a
century, from Burroughs and cybernetics to Derrida,~ Deleuze and
Dawkins, and turns *digital*media*junk*ware* into an updated version of
selfish cultural DNA - an information drug poisoning minds with
alluring rumors and spreadable beliefs. But the non-signifying discard
bites back: if meaningless genes cannot be selfish, how could
meaningless memes be?

*lore*\newline
\zh{无用之用} (\textit{wuyong zhi yong}, the
use of the useless)\newline
{\textquotedbl}How did this happen? How and when, exactly, did our
culture turn to junk? Or, in other words, when did we actually last
create some radically new cultural experiences? And when did we instead
start to recycle culture with the appearance (the glitter) or the
new?{\textquotedbl} (p. 169) - how many useless questions, Thierry!
Let{\textquotesingle}s tear them apart and build a new toolbox of
unusable concepts. If our entire culture turned to junk, it might be
the right time to go back to folklore. Junk refuses to be yet another
final statement in a series of grand narratives. Junk has to be found
in opposition to use value and organic garbage. Junk has to be rummaged
into for its discursive making, its affective collection, and its
temporary autonomous phases of repurposing into non-junk.
Let{\textquotesingle}s do away with both the myth of the original and
the myth of the copycat: [5C71?][5BE8?] \textit{shanzhai}
has gone all the way from junk into a culture into a product into an
industry into a rhetoric into a narrative and its hermeneutics - junk
for academics, the last rodents in line. Let{\textquotesingle}s stop
searching for our next brand of culture when the last choice is as
paradoxical as junk. Let{\textquotesingle}s drop our sampling devices
and drift along the kula rings of useless discard that we traverse and
traverse us at every corner of our mediated lives.
Let{\textquotesingle}s accept the loss of authorship and preservation,
of traceability and evolution, let{\textquotesingle}s just get high on
information drug, let{\textquotesingle}s believe and preach and sing
and dance in a carnival of *digital*media*junk*ware*lore*, a repertoire
of noise not-yet-useful to anyone, rumors without referent, fractal
jokes, non-human grotesques, post-linguistic vernaculars, unsuccessful
memes, self-censoring [6076?][641E?] \textit{egao}. Makers:
let{\textquotesingle}s {\textquotedbl}make do{\textquotedbl} rather
than {\textquotedbl}make to{\textquotedbl}.

*?*?*?*?*?*\newline
Sketched above are some mediations on a constellation of semantemes
recuperated from the reeky dustbin of recent media theory, in the
attempt of hacking together new hybrid concepts for a materialist
anthropology of the contemporary under the sign of
*digital*media*junk*ware*lore* or
*[6570?]*[5A92?]*[5E9F?]*[54C1?]*[4FD7?]*. In this light, moving from
Bardini{\textquotesingle}s bio-cultural parallels into speculative
probing of the *digital*, *junk* is understood as the Other of *ware*,
locked in J-W-J and W-J-W cycles of which we are the *media*, the
silent hyphens articulating a *lore* through each change of phase. The
best use to which we can put the signal-to-noise ratio of our genome is
unmasking the cultures to which we subscribe as ordered and stylized
collections of memes, a mere 1.5\% of the webs of meaning we code
without end. Praise the 98.5\%.
