\section{Acknowledgments}

We want to thank so much for their support our dear partner Sophia Lin,
from Make+ for her commitment and trust from the beginning of this
amazing adventure ! She has such a great and joyful energy and her very
professional and pragmatic has been very useful to us.

Catherine Ruggeri, general curator for the celebration of the
diplomatically relations between France and China, from Institut
Fran\c{c}ais, allowed us tho dream of this project and make it happen.~

Crossed lab want to thank also the cleverness and open minded and
adventurous Dicream for their fundamental support.

Junkware is an incredible human and spiritual adventure that allowed us
to open our minds and think abroad of our usual intellectual
boundaries.

~

\textbf{Make +\newline
}Founded in 2013 by artists, designers and engineers, Make+ helps
organize events and activities, so that these different communities can
exchange ideas and methods, and collaborate by working on real projects
together.[2028?][2028?]

website : \href{http://makeplus.org/}{makeplus.org}\newline
weixin: makeplus\newline
~

\textbf{CROSSED LAB\newline
}Crossed Lab is a production agency in digital arts. Crossing styles,
genres, and territories, Crossed Lab works on a renewed conception of
production. Placing them as an innovative writing incubator, Crossed
Lab~set up a hybrid place between artistic issues, research,
experimentation and transmission of our knowledge. They stick to a
contemporary process crossing bridges of practices and creative process
through a multimodal and trans-disciplinary approaches.

website :
\href{http://www.crossedlab.org/}{[2028?][2028?]crossedlab.org}\newline
vimeo :
[2028?]\href{http://vimeo.com/crossedlab}{vimeo.com/crossedlab}\newline
~

\textbf{WORKSHOP SUPPORT: One Space}~\newline
ONE is a creative experimental space located in Shanghai Jingan
district. It is open to all company and creative individuals.
Participants are welcomed to take part in creative thinking process and
space experimentations through workshops, talks and exhibitions.~

website : \href{http://one-magazine.net/one/}{one-magazine.net/one}


\bigskip

\textbf{Thierry Bardini Researcher\newline
}Author of the book Junkware published in 2011, Thierry Bardini
investigates~the origines and stakes of cyberculture.\newline
Trained as an agronomist, he is today a professor of sociology of
technologies at the University of Montreal, .

~

\textbf{Workshop participants}
Chen Yi, Feng Bin, Liu Feng , Jenny Lu , Albert Mayr, Han Yang , Miya Shen , Constance Garnier , Gabriele de Seta, Katy, Hao Rong, Li Ding, Jammes Jun Fang.


