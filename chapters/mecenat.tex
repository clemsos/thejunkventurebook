\section{Writing Machine, Algorithmic Papyrus \& Infinite Loops}

{\centering
\textit{Le texte ci-dessous \'etant susceptible
d{\textquoteright}\^etre repris sur les documents de communication de
Lorem Ipsum, nous vous remercions d{\textquoteright}\^etre le plus
pr\'ecis possible.}
\par}

Hiver 2104. La pluie battante s{\textquoteright}ab\^at sur le terminal
17 de la Gare Ouest de Shanghai. Les\newline
voyageurs prennent place \`a bord du vactrain SL58 qui les am\`enera 58
minutes plus tard en\newline
Gare de Lyon Part-Dieu. Attention \`a la fermeture des portes. Attention
au d\'epart.

Lyon-Shanghai : 58 minutes s{\textquoteright}interroge sur le futur des
transports \`a l{\textquoteright}\'echelle mondiale en donnant la
parole \`a ses usagers. LS58 propose \`a des habitants de Lyon et
Shanghai\newline
d{\textquoteright}imaginer l{\textquoteright}exp\'erience
d{\textquoteright}un monde o\`u un train relierai ces deux villes en 58
minutes seulement.\newline
Les participants expriment leurs visions au cours
d{\textquoteleft}ateliers d{\textquoteright}\'ecriture transm\'edia,
recr\'eant des \'el\'ements de cette r\'ealit\'e future. Une attention
particuli\`ere est port\'ee aux objets pr\'esents\newline
dans ce train, recr\'eer sous formes de mod\`eles en 3D. Gr\^ace \`a la
r\'ealit\'e augment\'ee et l{\textquoteright}impression 3D, ces objets
sont ensuite expos\'es comme autant de passerelles entre ce monde
futuriste virtuel et notre monde actuel. L{\textquoteright}univers
ainsi recr\'ee sera restitu\'e sous la forme d{\textquoteright}un
r\'ecit transmedia en ligne qui servira de prototype \`a un outil
d{\textquoteright}\'ecriture collaborative et multilingue pour supports
mobiles.

{\centering
* * *
\par}

{\centering
Mauris quam dolor, cursus at porta et, luctus eget purus. Nunc interdum.
\par}

{\centering
* * *
\par}

Lyon-Shanghai : 58 minutes. Au seuil de l{\textquotesingle}exp\'erience
sensorielle et digitale, LS58 propose d{\textquotesingle}entrouvrir une
porte pour vivre et construire ensemble le devenir technologique et les
futurs possibles de notre plan\`ete.

Franchir une simple porte pour vivre l{\textquotesingle}exp\'erience du
futur \`a l{\textquotesingle}autre bout de la plan\`ete,
c{\textquotesingle}est le pari du projet LS58. Gr\^ace aux
installations situ\'ees \`a Lyon et Shanghai, les spectateurs peuvent
s{\textquotesingle}immerger dans une r\'ealit\'e virtuelle d\'ecrivant
le futur de leurs villes, tel qu{\textquotesingle}imagin\'e par leurs
habitants.

LS58 brings together a community of artists, engineers and inventors in
France and China to\newline
imagine a (not so) far future where a train will be running between
Shanghai, China and Lyon,\newline
France in less than an hour. What will such fast transportation means to
daily life in both cities?\newline
Who will be the people using it and why? What will traveling in this
train feel like?

LS58 walks down the path of science-fiction to envision the future of
mobility and imagine how the shortening of path distance between places
and cultures will reflect in our everyday life.\newline
Using new technologies and digital art, LS58 happens in the forms of two
actual installations in\newline
Lyon and Shanghai. Bringing visions of the future to life, the
installations recreate an\newline
environment where connected objects and augmented reality merge into a
unique immersion\newline
experience into hereafter.

{\centering
* * *
\par}

{\centering
Mauris quam dolor, cursus at porta et, luctus eget purus. Nunc interdum.
\par}

{\centering
* * *
\par}

Lyon-Shanghai : 58 minutes Lyon-Shanghai : 58 minutes (LS58) is a
pluridisplinary project (LS58) between France and China that brings
together science, art and la Chine qui r\'eunit science, art design to
investigate the future of writing technologies.\newline
LS58 elaborate a collective representation of the future by generating
imaginary objects from visions and ideas expressed during workshops in
Lyon and Shanghai. Workshops participants provides blocks of experience
from their imagination to an algorithm which turns them narratives into
narrative forms describing objects. This anticipation work will be
released as a book-object in Fall 2014 during a conference and party
event in both Shanghai and Lyon.

In French, in English, in Chinese, in C++ or even in plastic, our
technological future uses everyday multiples langages to be written.
Science, stuck in a real-time becoming too present, has been sentenced
to the moment and therefore can{\textquoteright}t understand fiction
anymore. However, dreams that are floating everywhere still using a
future tense to be written. By focusing on anticipation writing, LS58
wants to offer an algorithmic papyrus to artists, designers,
technologists and enthusiats from France and China to play with words
and write down their ideas and wishes for tomorrow in today langages.

En fran\c{c}ais, en anglais, en chinois, en C++ ou m\^eme en plastique,
notre devenir technologique s{\textquoteright}\'ecrit chaque jour dans
de multiples langages. Condamn\'ee au moment, la science ne comprend
plus la fiction, englu\'ee dans un temps-r\'eel trop pr\'esent.
Pourtant, partout flottent des r\^eves qui s{\textquoteright}\'ecrivent
au futur. En s{\textquoteright}interrogeant sur le r\'ecit
d{\textquoteright}anticipation, LS58 veut offrir un papyrus
algorithmique \`a des designers, technologistes, artistes ou curieux de
France et de Chine pour jouer avec les mots et \'ecrire leurs id\'ees
pour demain en langages d{\textquoteright}aujourd{\textquoteright}hui. 

{\centering
* * *
\par}

{\centering
Mauris quam dolor, cursus at porta et, luctus eget purus. Nunc interdum.
\par}

{\centering
* * *
\par}

~

{\centering
Cher comit\'e de m\'ec\`enes,
\par}

~

Nous vous adressons cette lettre pour vous pr\'esenter
l{\textquoteright}\'evolution du projet artistique
\textit{Lyon-Shanghai : 58 minutes (LS58)}. Cette oeuvre
originale qui sera cr\'ee \`a Shanghai en Octobre prochain a m\^uri
depuis notre premi\`ere proposition. Nous tenons donc \`a vous exposer
ici les raisons et les avantages de ce changement.

~

Durant les derniers mois, les \'echanges avec nos partenaires en Chine
et l{\textquoteright}adaptation aux contraintes nous ont conduit \`a
renouveler notre vision du projet initial. Le travail de recherche
cons\'ecutif nous a amen\'e aux fronti\`eres de la narration et de
l{\textquoteright}exp\'erimentation technologique.

L{\textquoteright}id\'ee d{\textquoteright}une restitution unique sous
la forme de r\'ealit\'e virtuelle s{\textquoteright}est nourrie
d{\textquoteright}une r\'eflexion plus large sur le futur de
l{\textquoteright}innovation. Nous nous sommes donc orient\'es vers un
projet \`a plus long terme croisant science et fiction, o\`u se
m\^elent bio-technologies, analyse de donn\'ees et \'ecritures
g\'en\'eratives~

~

Pour traiter ces questions d{\textquoteright}avenir, nous avons
\'echang\'e avec notre partenaire Make+, pionnier et d\'efricheur dans
la d\'emocratisation et le d\'eveloppement des nouvelles technologies
\`a Shanghai. Ensemble, nous avons d\'ecid\'e que la mani\`ere la plus
pertinente de pr\'esenter ce travail de recherche dans des
\'ev\`enements disposant d\'ej\`a des couvertures m\'ediatiques et
attirant des professionnels et un public averti.

Le Maker Carnival et le DAFF (Design Art \& Fashion Fair) r\'eunissent
sur deux week-ends quelques 30.000 visiteurs et professionnels venus du
monde entier. Ev\`enements cl\'es de la sc\`ene culturelle de Shanghai,
ils nous permettent de rencontrer des publics vari\'es et un r\'eseau
m\'ediatique plus influent, constitu\'e tant de blogueurs suivis que de
journalistes d{\textquoteright}envergure.

Cette forme aboutie du projet re\c{c}oit des encouragements et des
soutiens marqu\'es, \`a commencer par celui du Centre National du
Cin\'ema via le Dispositif d{\textquoteright}Aide \`a la Cr\'eation
Num\'erique (DICREAM) \`a hauteur de 15 000 euros ce qui nous laisse
optimiste quant \`a l{\textquoteright}avenir de ce projet.

L{\textquoteright}innovation ouverte devient aujourd{\textquoteright}hui
une question centrale aux diff\'erentes activit\'es humaines. En
s{\textquoteright}appuyant sur un r\'eseau pluri-disciplinaire et
international d{\textquoteright}id\'ees et de pratiques cr\'eatives
(science, \'economie, technologie, litt\'erature, art...),
\textit{LS58 }devenu \textit{Junkware}
b\'en\'eficie d{\textquoteright}une perspective unique et novatrice sur
le futur des relations entre les peuples et leur capacit\'e \`a cr\'eer
ensemble.


\bigskip

Nous vous remercions par avance de votre soutien renouvel\'e \`a ce
projet.

~

Cl\'ement Renaud

\& l{\textquoteright}\'equipe du projet Junkware

~

{\centering
* * *
\par}

~

D\'evelopp\'e entre Lyon et Shanghai, Junkware est un projet
exp\'erimental pluridisciplinaire sur l{\textquotesingle}\'ecriture
collective du futur des technologies dans nos vies. En fran\c{c}ais, en
anglais, en chinois, en C++ ou en plastique, notre devenir
technologique s{\textquoteright}\'ecrit chaque jour sur nos corps dans
de multiples langages. Junkware propose une rencontre incongrue de ces
diff\'erents {\textquotedbl}codes{\textquotedbl} dans une r\'eflexion
crois\'ee sur le r\^ole de l{\textquotesingle}innovation technologique
et la production du futur.

~

D\'evelopp\'e par Junkware Inc., le Invention Sequencer est une machine
\'etrange permettant de g\'en\'erer des objets futuristes \`a
l{\textquotesingle}aide d{\textquotesingle}un fragment
d{\textquotesingle}ADN (cheveu, salive...) et des ondes \'emises par le
cerveau en utilisant la mod\'elisation math\'ematique et
l{\textquotesingle}analyse de donn\'ees : une exp\'erience unique et
extra-ordinaire. L{\textquotesingle}installation Junkware propose de
cr\'eer et de montrer une galerie d{\textquotesingle}objets imagin\'es
et d\'evelopp\'es ensemble par l{\textquotesingle}homme et la machine.

Junkware is a French-Chinese inter-displinary project that brings
together science, art and design to investigate the future of life with
technologies.Developped between Lyon and Shanghai, Junkware is an
experimental pluri-disciplinary project about collective writing of
futur of technologies in our lives.

In French, in English, in Chinese, in C++ or even in plastic, our
technolocal becoming is written every day on our bodies by multiple
languages. Junkware offers an unexpected encounter of those differents
{\textquotedbl}codes{\textquotedbl} in a reflection on the role of
technological innovation in the production of the future.\newline
Created by Junkware Inc., the Invention Sequencer is a strange machine
that uses mathematical modelisation and data analysis to generate
futuristic objects from a fragment of your DNA (hair, saliva...) and
the waves emitted by your brain. The Junkware installation offers to
create and show a gallery of objects imagined together by the human and
the machine.\newline
\newline
