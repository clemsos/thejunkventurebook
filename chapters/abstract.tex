\textbf{Abstract}\\
Junkware brings together theater and generative writing to engage a reflection about the making of technological innovation. In the installation, the audience is invited to input biometrics data into a machine called Junk Sequencer that recreates a visual and textual description representation of a futuristic and fictional object by matching a database of patents with samples of their DNA and neural activity obtained with an electroencephalographic (EEG) headset. Fictional characters representing the authority figure of science, the Pr. Lafleche, and his team of assistants invites the audience to become guinea pigs and test the machine to receive a junk honorary credential as compensation for their participation. Created for professional design fairs in Shanghai (China), this digital farce exposes the absurdity of the mechanisms behind scientific and technological innovation while leading a formal reflexion about the use of generative writing in the context of performance arts.



\textbf{Keywords}\\
generative writing, theater, absurd, grotesque, patents, innovation, China, DNA sequencing, EEG
